\documentclass[a4paper,12pt]{article}

\usepackage[german]{babel}
\usepackage{url}
\usepackage{hyperref}
\usepackage[nameinlink,noabbrev]{cleveref}
\usepackage{multirow}
\usepackage{lipsum}
\usepackage{enumitem}
\usepackage[utf8]{inputenc}
\usepackage{marvosym}

\setlist[enumerate]{itemsep=0mm}

\author{Ivan Kolomiiets \and Muhammad Ihsan Ali Akbar}
\title{Arbeitslosenversicherung: Handout}

\begin{document}
	\maketitle
	\begin{abstract}
		Das vorliegende Dokument offeriert eine kompakte Übersicht über die Thematik der Arbeitslosenversicherung. Im Fokus der Betrachtung stehen dabei die historische Entwicklung, die Träger der Versicherung, die Funktionsweise, die angebotenen Leistungen, die Pflichten der Teilnehmenden sowie assoziierte Themengebiete.
	\end{abstract}
	\tableofcontents
	\section{Geschichte}
	\begin{enumerate}
		\item Arbeitslosenversicherung findet ihre gesetzliche Grundlage im Dritten Buch Sozialgesetzbuch (SGB III) \cite{Wikipedia_contributors_undated-jq}.
		\item Die gesamte Sozialgesetzgebung geht auf den Reichskanzler Otto von Bismarck (1815 – 1898) zurück \cite{Wikipedia_contributors_undated-jq}.
		\item Ziel der Arbeitslosenversicherung war die soziale Not der Arbeitnehmerschaft Mitte des 19. Jahrhunderts zu lindern \cite{noauthor_2021-mg}.
		\item Die Arbeitslosenversicherung in Deutschland wurde am 16. Juli 1927 durch das Gesetz über Arbeitsvermittlung und Arbeitslosenversicherung eingeführt und der Reichsanstalt für Arbeitsvermittlung und Arbeitslosenversicherung übertragen \cite{noauthor_2021-mg,Wikipedia_contributors_undated-jq}.
		\item Der Beitrag zur Arbeitslosenversicherung wurde am 01. Januar 2020 auf 2,4\% per Verordnung festgelegt \cite{noauthor_2021-mg,Hery-Mosmann2020-rb}. Seit 1. Januar 2023 auf 2,6\% \cite{Wikipedia_contributors_undated-jq}. Siehe \autoref{tab:time-percent}.
	 	\begin{table}[h]
			\centering
			\begin{tabular}{ |l|l| }
				\hline
				Zeitraum & Beitragssatz \\
				\hline
				1992 & 6,2\% \\
				\hline
				1993-2006 & 6,5\% \\
				\hline
				2007 & 4,2\% \\
				\hline
				2008 & 3,3\% \\
				\hline
				2009 und 2010 & 2,8\% \\
				\hline
				2011-2018 & 3,0\% \\
				\hline
				\multirow{1}{*}{2019} & \multirow{1}{*}{} \\
				laut Gesetz: & 2,6\% \\
				per Verordnung: & 2,5\% \\
				\hline
				\multirow{1}{*}{2020-2022} & \multirow{1}{*}{} \\
				laut Gesetz: & 2,6\% \\
				per Verordnung: & 2,4\% \\
				\hline
				seit 2023 & 2,6\% \\
				\hline
			\end{tabular}
			\caption{``Arbeitslosenversicherung in Deutschland``, \cite{Wikipedia_contributors_undated-jq}. \emph{Beitragssatz zeigt ein Beitrag der Arbeitnehmer und Arbeitgeber insgesamt.}}
			\label{tab:time-percent}
		\end{table}
		\item In der Arbeitslosenversicherung steigt die Beitragsbemessungsgrenze nach den BMAS-Vorgaben und gängigen Übersichten für 2026 auf \texteuro101.400 pro Jahr beziehungsweise \texteuro8.450 im Monat \cite{Klose2025-kp}. D.h., wenn man Milliarden verdient, werden 1.3\% nicht von Milliarden abgerechnet, sondern nur von \texteuro8.450.
	\end{enumerate}
	\section{Träger}
	\begin{enumerate}
		\item Träger der Arbeitslosenversicherung ist die Bundesagentur für Arbeit in Nürnberg \cite{Wikipedia_contributors_undated-jq}. 
		\item Aufsichtführendes Ministerium ist das Bundesministerium für Arbeit und Soziales \cite{Wikipedia_contributors_undated-jq}.
		
		Bezüglich der Finanzierung ist eine Beteiligung von Arbeitnehmern und Arbeitgebern in gleichem Maße vorgesehen, wobei auch die Möglichkeit von Zuschüssen besteht \cite{noauthor_2021-mg}. Dies impliziert, dass die 2,6\% der Beiträge, die zum Zeitpunkt der Erstellung dieses Handouts aktiv sind, zu gleichen Teilen auf Arbeitnehmer und Arbeitgeber aufgeteilt werden: jeweils 1,3\% pro Partei.
	\end{enumerate}
	\section{Funktionsweise}
	\begin{enumerate}
		\item Von den Leistungen der Arbeitslosenversicherung profitieren Arbeitnehmer, wenn sie zuvor in die Versicherung eingezahlt haben und arbeitslos werden \cite{noauthor_2021-mg}.
		\item Arbeitslosenversicherung einspringt nur, wenn der Arbeitnehmer von seinem Arbeitgeber gekündigt wurde. Hat er selbst das Arbeitsverhältnis beendet oder einer Vertragsauflösung zugestimmt, kann er keine Ansprüche aus der Arbeitslosenversicherung geltend machen \cite{noauthor_2021-mg}. 
		\item Für die Finanzierung tragen Arbeitnehmer und Arbeitgeber jeweils ihren Anteil, der bei der Erstellung der Lohnabrechnung berücksichtigt wird \cite{noauthor_2021-mg}.
		\item Arbeitnehmer, die ein bestimmtes Jahreseinkommen (Grenzwert, \texteuro8.450 im Monat für Arbeitslosenversicherung ab 2026) überschreiten, müssen nur bis zu einem gewissen Betrag Sozialversicherungsabgaben leisten \cite{noauthor_2021-mg}.
	\end{enumerate}
	\section{Leistungen}
	\begin{enumerate}
		\item Das Abgeben von Versicherten Teile ihres Arbeitseinkommens, um Vorsorge für den Fall zu treffen, in dem sie kein eigenes Einkommen erzielen. 
		\item Die Arbeitslosenversicherung schützt vor finanziellen Folgen einer Arbeitslosigkeit.
		\item Arbeitslosenversicherung hat das Ziel, arbeitssuchenden Personen während ihrer Arbeitssuche das Einkommen zu sichern.
		\item \emph{Berufsberatung und Arbeitsmarktberatung} \cite{Wikipedia_contributors_undated-jq}
		\item \emph{Ausbildungsvermittlung und Arbeitsvermittlung} \cite{Wikipedia_contributors_undated-jq}
		\item Leistungen zur Aktivierung und beruflichen Eingliederung
		\begin{enumerate}
			\item z. B. Förderung aus dem Vermittlungsbudget
		\end{enumerate}
		\item Leistungen zur Berufswahl und Berufsausbildung
		\begin{enumerate}
			\item Berufsorientierungsmaßnahmen
			\item Berufsvorbereitende Bildungsmaßnahmen
			\item Berufsausbildungsbeihilfe
		\end{enumerate}
		\item \emph{Leistungen zur beruflichen Weiterbildung} \cite{Wikipedia_contributors_undated-jq}
		\begin{enumerate}
			\item Übernahme von Weiterbildungskosten
		\end{enumerate}
		\item Leistungen zur Aufnahme einer Erwerbstätigkeit
		\begin{enumerate}
			\item Eingliederungs- und Gründungszuschuss
		\end{enumerate}
		\item Leistungen zum Verbleib in Beschäftigung
		\begin{enumerate}
			\item Kurzarbeitergeld, Saison-Kurzarbeitergeld, Transferkurzarbeitergeld
		\end{enumerate}
		\item Leistungen der Teilhabe behinderter Menschen am Arbeitsleben
		\item \emph{Arbeitslosengeld, Teilarbeitslosengeld, Arbeitslosengeld bei Weiterbildung und Insolvenzgeld} \cite{Wikipedia_contributors_undated-jq}
	\end{enumerate}
	Anspruch auf Arbeitslosengeld bekommt der, der
	\begin{enumerate}
		\item ohne Beschäftigung ist,
		\item kann aber mindestens 15 Stunden pro Woche einer versicherungspflichtigen Beschäftigung nachgehen,
		\item eine Stelle sucht, die versicherungspflichtig ist,
		\item Anwartschaftszeit erfüllt \cite{arbeitsagenturArbeitslosengeldAnspruch}.
	\end{enumerate}
	\subsection{Anwartschaftszeit}
	Anwartschaftszeit ist erfüllt, wenn Arbeitslosmelder in den \emph{30 Monaten} vor der Arbeitslosmeldung und Arbeitslosigkeit in der Arbeitslosenversicherung mindestens \emph{12 Monate} pflicht- oder freiwillig versichert war.
	In der Regel werden versicherungspflichtige Zeiten in Beschäftigungsverhältnissen zurückgelegt.
	Zur Berechnung, ob die Anwartschaftszeit erfüllt ist, werden die Zeiten \emph{aller versicherungspflichtigen Beschäftigungen innerhalb des 30-Monate-Zeitraumes zusammengerechnet} \cite{arbeitsagenturArbeitslosengeldAnspruch}.
	
	\subsubsection{Verkürzte Anwartschaftszeit bei befristeten Beschäftigungen}
	War Arbeitslosmelder häufig befristet beschäftigt, gilt unter bestimmten Voraussetzungen eine kürzere Anwartschaftszeit. Dann genügt es, wenn man auf 6 Monate oder mehr versicherungspflichtige Zeiten kommt (Beschäftigung oder weitere versicherungspflichtige Zeit) in den 30 Monaten vor der Arbeitslosmeldung und Arbeitslosigkeit \cite{arbeitsagenturArbeitslosengeldAnspruch}. 
	
	Voraussetzung dafür ist unter anderem, dass die im 30-Monats-Zeitraum überwiegend ausgeübten Beschäftigungen im Voraus auf höchstens 14 Wochen befristet waren und das Arbeitsentgelt der letzten 12 Monate einen bestimmten Wert nicht überschreitet \cite{arbeitsagenturArbeitslosengeldAnspruch}.
	
	\subsection{Höhe des Arbeitslosengeldes}
	Die Grundlage, auf der Arbeitslosengeld berechnet wird, ist das Brutto-Arbeitsentgelt (Gehalt) der vergangenen 12 Monate \cite{arbeitsagenturArbeitslosengeldAnspruch}. 
	
	Dabei wird nur der Teil des Arbeitsentgelts berücksichtigt, \emph{der beitragspflichtig in der Arbeitslosenversicherung war} (also zum Beispiel kein Minijob) und beim Ausscheiden aus dem Beschäftigungsverhältnis abgerechnet war. Indem der Betrag durch 365 geteilt wird, wird das Brutto-Arbeitsentgelt pro Tag ermittelt. Es wird als \emph{Bemessungsentgelt} bezeichnet. Davon werden rein rechnerisch die Lohnsteuer, gegebenenfalls der Solidaritätszuschlag und ein Pauschalbetrag für die Sozialversicherung in Höhe von \emph{20 Prozent} abgezogen. Das Ergebnis ist das Netto-Entgelt pro Tag, das als \emph{Leistungsentgelt} bezeichnet wird. 
	
	\emph{60 Prozent des Leistungsentgelts} sind der Betrag, den Arbeitslosmelder als Arbeitslosengeld pro Tag erhält \cite{arbeitsagenturArbeitslosengeldAnspruch}. Dieser Betrag erhöht sich auf \emph{67 Prozent}, falls Arbeitslosmelder oder der Ehe-/ Lebenspartner des Arbeitslosmelders mindestens ein Kind (im Sinne des Einkommenssteuergesetzes) hat \cite{arbeitsagenturArbeitslosengeldAnspruch}.
	
	\subsection{Dauer des Arbeitslosengeldes}
	Für das Berechnen von dem Dauer des Arbeitslosengeldes, müssen die versicherungspflichtigen Zeiten in der um 30 Monate verlängerten Rahmenfrist, das heißt, \emph{in den vergangenen 5 Jahren} liegen. Dabei werden mehrere versicherungspflichtige Zeiten zusammengerechnet \cite{arbeitsagenturArbeitslosengeldAnspruch}.
	\subsubsection{Anspruchsdauer für Arbeitslose bis 50 Jahre}
	Ist Arbeitslosmelder jünger als 50 Jahre, kann er oder sie höchstens für die Dauer von 12 Monaten Arbeitslosengeld erhalten – vorausgesetzt, man war zuvor 24 Monate oder länger versicherungspflichtig \cite{arbeitsagenturArbeitslosengeldAnspruch}. 
	
	Weiteres Beispiel: War man 12 Monate versicherungspflichtig, erhält man 6 Monate Arbeitslosengeld.
	\subsubsection{Anspruchsdauer für Arbeitslose ab 50 Jahre}
	Ab dem vollendeten 50. Lebensjahr steigt die Anspruchsdauer in mehreren Schritten auf bis zu 24 Monate an. Diese höchste Anspruchsdauer gilt für Arbeitslose, die \emph{58 Jahre oder älter sind}. Voraussetzung ist, dass man 48 Monate oder länger versicherungspflichtig war.
	\subsubsection{Anspruchsdauer bei kurz befristeten Beschäftigungen}
	Wenn man die Voraussetzungen für die verkürzte Anwartschaftszeit erfüllt, gilt: Kommt man zum Beispiel auf 8 versicherungspflichtige Monate, erhält man 4 Monate Arbeitslosengeld.
	\section{Pflichten}
	Rechtsgrundlage der Arbeitslosenversicherung ist das SGB III ``Arbeitsförderung``. Pflichtversichert sind vor allem Personen, die gegen Arbeitsentgelt mehr als geringfügig (Minijobs, geringfügige Beschäftigung) beschäftigt oder in Berufsausbildung sind (§ 25 SGB III) \cite{noauthor_undated-hv}.
	\begin{enumerate}
		\item Grundsätzlich müssen alle sozialversicherungspflichtigen Beschäftigten ihren Beitrag zur Sozialversicherung leisten \cite{noauthor_2021-mg}.
		\item Ein sozialversicherungspflichtiger Arbeitnehmer arbeitet in einem Arbeits- oder Angestelltenverhältnis \cite{noauthor_2021-mg}.
		\item Auch die Arbeitgeber müssen sich zur Hälfte an der Sozialversicherung ihrer Arbeitnehmer beteiligen \cite{noauthor_2021-mg}.
	\end{enumerate}
	Von der grundsätzlichen Regelung gibt es zwei Ausnahmen \cite{noauthor_2021-mg}:
	\begin{enumerate}
		\item Beschäftigungsverhältnisse, bei denen der Arbeitnehmer nicht mehr als \texteuro556 verdient, sind sozialversicherungsfrei (Minijob).
		\item Beschäftigungsverhältnisse, bei denen der Arbeitnehmer zwischen \texteuro556 und \texteuro2000 erzielt (Midijob). Der Arbeitnehmer zahlt nicht die vollen Beiträge zur Sozialversicherung.
		\begin{enumerate}
			\item Midijob: bei \texteuro556 --- 28\%, sinkt bis \texteuro2000 auf 20\%.
		\end{enumerate}
	\end{enumerate}
	Von der Sozialversicherungspflicht ausgenommen sind die folgenden Berufsgruppen \cite{noauthor_2021-mg}:
	\begin{enumerate}
		\item Selbstständige und Unternehmer
		\item Beamte
	\end{enumerate}
	Selbstständige und Unternehmer und Beamte können sich aber auf freiwilliger Basis gegen Arbeitslosigkeit versichern \cite{noauthor_2021-mg}.
	\bibliographystyle{plain}
	\bibliography{references}
\end{document}